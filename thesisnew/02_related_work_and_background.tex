\chapter{Related Work and Background}
\label{ch:Related Work and Background}
If one takes a look at existing work in the field of intelligent parking systems, he will encounter countless scientific papers in the most diverse sub-areas of intelligent parking systems. In addition, there are numerous projects in which such systems have already been implemented. There are also some works that have set themselves the task of categorizing related work in this field. Nevertheless, there does not seem to be any definite consensus among scientists on definitions and categorizations. In the following we will present an introduction to the topic based on these papers.

\section{Overview}

\subparagraph{Intelligent Parking Systems}
In general terms, Intelligent Parking Systems, also called smart parking systems, are systems that try to solve the countless parking problems already mentioned while integrating 'advanced technologies and researches from various academic disciplines' \cite{idris2009car}. Thus the words intelligent or smart have different meanings depending on the time at which a system was developed \cite{fraifer2016investigation}. With further progress, especially in information technology, new opportunities are opening up and old implementations no longer seem to be 'advanced' or 'smart', but were at the time of development.
\subparagraph{Shared Parking}
Shared Parking Systems form a separate category of intelligent parking systems, because they use the emerging 'shared technology' in connection with mobile devices, which are available to most people due to technological advances. Shared parking provides the framework for making purely private parking spaces available to the public by allowing them to be traded as a commodity. \cite{itdp2014shared} By sharing, parking spaces can be used more effectively and information can be distributed more easily, which reduces the aforementioned parking problems. It is taken advantage of the fact that most private parking spaces are only used by certain groups of motorists at particular times and thus remain unused for a large part of the time and that those parking times for the different groups often do not overlap.\cite{vtpi2015shared} A supermarket, for example, needs its parking spaces mainly during the day when the supermarket is opened, while a hotel needs the parking spaces at night for the overnight guests.\\
\section{Intelligent Parking Systems}
To give an overview for intelligent parking systems we are introducing a categorization of those systems proposed by Susan Shaheen in Smart parking management field test: A bay area rapid transit (bart) district parking demonstration.\cite{shaheen2005smart} Shaheen divides intelligent parking systems into five categories.
\subparagraph{Guidance Information Systems}
Guidance information systems are the simplest type of intelligent parking systems. The aim of such systems is to point the car park seeker to the nearest available parking spaces. In many cities this is done by simple displays on the streets, which point to car parks and show the number of free parking spaces available in real time.

\subparagraph{Transit-Based Information Systems}
Transit-base information systems are just like the previously mentioned GIS pure information systems. However, special importance is given in this case to draw attention to parking spaces and to direct to parking spaces that are linked directly to local public transport. In most cases such parking spaces are called 'Park+Ride' and allow free parking, for example. The main objective of these systems, in addition to reducing the traffic of those seeking parking, is to reduce any traffic in the city centres. The aim is to make public transport more attractive and useable in order to prevent air pollution and congestion in the cities themselves. 

\subparagraph{Smart Payment Systems}
With smart payment systems operators try to minimize their maintenance costs. Setting up, emptying and repairing coin-operated machines at parking lots is often an expensive business. Operators of Smart Payment Systems therefore try to forgo using these machines. Instead, the customer pays, even contactless, with smartcards or via his own smartphone. This usually has advantages for both sides. For the customer the payment process is accelerated and he can avoid using cash and at the same time the operating costs are reduced.

\subparagraph{E-parking}
By e-parking, Shaheen referred to a special system that was under development in 2005 by a research consortium. The aim of the concept was to combine the various smart parking ideas. The system includes advanced guidance information systems, smart payments methods, reservation options and connectivity to other digital services. More generally, we consider e-parking to be systems that use all means of digitalisation to make the parking experience better and more effective for the user in all possible aspects.

\subparagraph{Automated Parking}
Automatic parking is a system in which the user simply drives the car to a general drop-off point and later picks it up from there. The system takes over the remaining transport route of the vehicle from the delivery point to the actual parking lot. Such systems also exist as manual versions, then called valet parking.

\section{Shared Parking}
Shaheen, like many others, pays little to or no attention to the topic of shared parking. However, according to the above definition, shared parking is a kind of intelligent parking system. Overall, there are hardly any scientific papers on this subject. There are already a few implementations of the idea, but these are either limited to certain cities or have hardly any users, which renders them useless. In addition, these systems are operated by private companies and therefore no information about the security within the system is publicly available.\\

Of the few scientific papers that exist, most are concerned with the allocation and pricing of parking spaces to potential tenants.

Shao et al. \cite{shao2016simple} and Yu et al. \cite{yu2018optimal} propose to use integer linear programming to maximize profits. Both studies show that their proposed strategies, in contrast to 'first-come-first-serve' or 'first-book-first-serve' strategies, lead to higher parking space utilisation and greater profits for operators of shared parking systems. Yu et. al. refer in their work only to the sharing of parking spaces of shopping malls at night. This also implies that, unlike our system, there is no fluctuation in the number of parking spaces over time which their system would not be able to handle.

Xu et al. \cite{xu2016private} extend a classic matching mechanism (top trading cycles, ttc) for the parking spaces so that cash flow is allowed. They assumed that each participant owns at least one parking space and participates with this parking space in the ttc mechanism. For situations where participants are unable to rent out their own parking space or do not wish to rent another parking space, the mechanisms '(price-compatible) top trading cycles and deals (TTCD)' and 'price-compatible top trading cycles and chains (PC-TTCC)' are proposed as extensions to ttc. These allow, in addition to the already possible exchange of parking space in ttc, the flow and exchange of money within the system. These mechanisms allowed participants to save costs, but under certain circumstances negative platform's payoff would occur. 

Xiao et al. \cite{xiao2018shared} assign parking spaces to tenants in their work using a double auction mechanism. They present the "demander competition padding method (DC-PM)" auction mechanism based on a slot allocation rule and a transaction payment rule. Both demanders and bidders place their bids by specifying the parking price and time and the mechanism determines the allocation of the parking spaces. Since this mechanism can lead to distorted social welfare, a "modified demander competition padding method (MDC-PM)" auction mechanism is also proposed, which is however inferior to the first mechanism in terms of its usefulness for the participants. They show that both mechanism "can realize asymptotic efficiency as both demanders and suppliers approach infinity", but do not take the spatial requirements of the participants into account during the allocation process. 

Zhang et al. \cite{zhang2019pricing} try to improve the competitiveness of shared parking lot providers compared to traditional parking lots. They use the Hotteling model to compare the product differentiation of the two parking systems and use equilibrium analyses to evaluate the parking prices. 

In all these papers, the utilisation of parking spaces or the social welfare in the system improves compared to the 'first-book-first-serve' mechanism, but the systems do not provide the user with the same level of transparency. In our shared parking system, users should be able to see exactly which parking spaces are available, choose from them and be sure to be able to use the selected parking space. This can only be implemented with 'first-book-first-serve' and the systems mentioned above are not able to provide this functionality.\\

In addition to parking space allocation through a matching mechanism, Kim et al. \cite{Kim2015ASP} propose an infrastructure to help drivers find the optimal parking space. The infrastructure consists of Road Side Units (RSUs), which communicate with the cars of parking seekers and Fog servers, which monitor the availability of each parking lot. RSUs communicate directly with Fog servers in their vicinity and with each other through a Roadside Cloud. Users can now make a request while driving, which is then accepted by the RSUs. Through communication between the RSUs and with the Fog Servers, a matching problem with all car park seekers is established, solved and the allocation of parking spaces is sent back to the vehicles through the RSUs. 

The big problem for this system are the expensive implementation and maintenance costs which arise because a lot of hardware and sensors have to be installed. The operator has to provide the RSUs, the Fog Server, sensors for each parking spot and the Roadside Cloud and every participant requires appropriate channels of communication on his car to connect to the RSUs. Our system, on the other hand, will do without expensive hardware to reduce maintenance costs and enable anyone to participate.\\


A number of shared parking platforms have already been implemented, but in very few cases they have become established. Internationally, for example, justpark.com or mobypark.com offer this service, in Germany there are providers like parkplace.de and parklist.de. All these platforms offer private individuals the opportunity to offer their own parking spaces, but the supply is very limited.

At parkinglist.de, even in large German cities such as Munich, there is almost exclusively only information about public car parks or offers for monthly rented parking spaces from car park operators, but hardly any private offers.

At parkplace.de there is also only a small number of almost exclusively long-term parking spaces available. Moreover, this platform does not offer the possibility to automatically perform payments. Instead, payment-specific matters have to be agreed on with the landlord separately.

At justpark.com, which originates from England, there are almost exclusively public parking garages available in London, for example. However, many car parks also offer the possibility of making reservations.

At mobypark.com you can find some listings from private persons. However, Mobypark is not widespread and is only really represented in a few cities, such as Amsterdam and Paris.

However, none of the providers discloses information on how they can protect their systems and users from fraud and how they generally deal with fraud. This is presumably because, as Mobypark itself reports on its website, parking spaces are hardly rented for short periods of time like a few hours, but mainly for longer periods (several days to several months). In addition, most of the rented parking spaces are not publicly accessible and key cards or the like must be exchanged prior to and at the end of the rental process. These conditions already lead to a reduction of fraud possibilities. In our application, however, it should also be possible to rent freely accessible parking spaces, even for short periods of time. This creates new opportunities for fraud, which, unlike in the scientific papers mentioned above, we will address in this paper.

\section{Trust Systems}\label{sec:Trust Systems}
In order to distinguish malicious and genuine users in our system, we will use a trust system. Since there is a large amount of scientific work and discussion in the area of trust systems, we will not develop any new methods ourselves and will instead fall back on already matured methods, adapt them to our system and thereby make them better fit our system.\\

Trust systems are used in many different forms for many different purposes. Therefore, we have sought to find methods developed for systems that are similar to ours. Finally, we will use a method from the area of "Trust Management and Reputation Systems in Mobile Participatory Sensing Applications". These are systems in which participants use smartphones or similar devices to send information that they received at various locations through the sensors of their devices to a backend server. We will later show how trust methods of such systems can be easily mapped to our trust system. Mousa et. al. wrote a survey on this topic in 2015, which gives a fundamental overview of this area of trust systems\cite{mousa2015trust}:\\

In all these systems, trust is used to assess the reputation of individual users. The calculated reputation is therefore a value for the trustworthiness of the participants and the quality of their contributions. Mousa classifies trust/reputations systems according to methodology, type of distribution and anonymity of participants. 

\subparagraph{Methodology} Either a Trusted Platform Module (TPM) or a reputations score can be used as a methodology for trust assessment. TPM are hardware chips that are installed in the devices of the participants and are therefore not suitable for our system. In the "Future Work" section, trust assessment using TPMs is discussed briefly. For our system, on however, the use of a reputation score, which is calculated directly from the behavior of the participant, is suitable. For the computation of this score the participants' latest activities and his old score are used.

\subparagraph{Type of distribution} The distribution in a trust system can be either central, collaborative or a combination of both. Since our trust score should be hidden from the participants and we possess a trust server, a centralized structure is suitable for us. In this case the trust server receives all information, i.e. the contributions and feedback of the participants, from which it is able to calculate the score of all participants. In collaborative systems, however, each participant computes a separate score for all other participants.

\subparagraph{Anonymity of participants} Finally, anonymous and non anonymous systems can be distinguished. In anonymous systems the anonymity of the participants is preserved. This is neither necessary nor useful in our system, as each participant must provide certain personal data to access our system. \\

In the section x.y.z.z reputation module we will use a trust system corresponding to the above established classification and modify it for our purposes.

The use of such a trust system however allows different attack types which are described in more detail in section x.y attacker model. In the security analysis we will finally show the robustness of our system against these attacks.
