\chapter{Introduction}
\label{ch:Introduction}
Parking in large cities is a well-known problem. For decades, the search for a parking space has been something that annoys most people. Due to the ever faster growing cities in recent years, the resulting difficulties are only intensifying. The ensuing traffic for example already represents a significant part of the total traffic in city centres. According to studies around 30 percent of the daily city traffic comes from car park seekers\cite{shoup2006cruising}. In addition to the rising number of people in cities, the percentage of people who own a car continues to increase, which adds to the traffic problems. However, the traffic is not just frustrating for the drivers. The ensuing traffic jams and the slow-moving traffic cause an increased amount of petrol to be burned and the air to become additionally polluted. Especially in these times, when air pollution and climate warming are becoming even more important due to Dieselgate and the capricious weather conditions, one should be looking for a solution.

The creation of new parking spaces is rarely an option. On one hand, this is often associated with high costs, on the other hand, there is usually no space for the construction of new parking spaces available in the cities. Instead, existing parking spaces must be used more effectively.

This can be done in many different ways. So-called intelligent parking systems increase the smartness of parking in different areas and therefore often lead to a more effective use of parking spaces. Intelligent parking systems are available in various forms and there are numerous scientific papers and completed projects for most of them. However, there are hardly any relevant developments for shared parking systems, which are by definition a type of intelligent parking system. A shared parking system is a trading platform on which private users can rent out and rent parking spaces. Sharing systems, such as car sharing, have become increasingly influential in recent years and are expected to continue to grow strongly \cite{freese2014shared}. If an object is shared among different users, it is used more effectively. This applies to parking places and a shared parking system as well. Therefore, the main part of this work will be concerned with designing a shared parking system. Special attention is paid to security aspects by detailing how the system handles fraud.

First, an overview of the system is presented, in particular the system model, which depicts the different roles in our shared parking system. This is followed by functional and security requirements to be met by the design. In contrast to existing systems, the design of our system will try avoid complex hardware and manual intervention by the service provider. In addition, this work deals in particular with fraud in a shared parking system and ways to deal with it. Possible cases of fraud within the system are described and grouped into different classes of fraud. It is noted that our system must be able to recognize and prevent all of them, punish adversaries and compensate injured parties.

The actual design is described in the next step. First all functional features and then the security features that determine how to deal with fraud are explained. We will introduces various modules that help dealing with fraud. There is a rating module, a module for reporting adversaries, a module to help verify fraud cases and a reputation module, which assigns a score to each user that indicates how trustworthy that user is.

Subsequently, the design is evaluated on the basis of the requirements. We shall find, that our system is robust against all the attacks listed before. The shared parking system therefore complies with the specified security requirements and can successfully deal with all classes of fraud cases.

Once it has been shown that the design meets the requirements, an implementation of a prototype, which is also part of this study, will be discussed in more detail. The design was implemented as a client-server model and includes a Java web service, which serves as an interface to a central database, and an Android app for the end user. All three parts are described in detail and it is explained how to execute the attached code and get the prototype up and running. At the end of the chapter, some aspects, which could be applied to help with the realization of this shared parking system, are pointed out.

Finally, a number of future work topics are presented which were not processed in this study. In the future, for example, alternative reputation systems could be applied and evaluated or the performance and good usability of the app could be proven.

At the end, a summary of the achieved results is provided.